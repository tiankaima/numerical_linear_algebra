\documentclass{article}
\usepackage[UTF8]{ctex}
\usepackage{hyperref,float,indentfirst,verbatim,fancyhdr,graphicx,listings,longtable,amsmath, amsfonts,amssymb}
\newenvironment{metaverbatim}{\verbatim}{\endverbatim}

\textheight 23.5cm \textwidth 15.8cm
\topmargin -1.5cm \oddsidemargin 0.3cm \evensidemargin -0.3cm

\title{数值代数实验报告 8}
\author{马天开}

\begin{document}
\maketitle

\section{问题描述}

\subsection{SVD 迭代}

参考课本 7.6.2 节 (P234-240)SVD 迭代完成 SVD 算法 7.6.3, 并对附件 svddata.txt 中的矩阵作 SVD 分解 $A = P\Sigma Q$。并计算 $PP^T-I, QQ^T-I, P\Sigma Q-A$ 的绝对值最大的元素,依次用 ep, eq, et 表示。

\begin{metaverbatim}
输出格式为:
迭代次数:x
奇异值从小到大:
ep = xx
eq = xx
et = xx

A=PTQ
T= [矩阵]
P= [矩阵]
Q= [矩阵]
\end{metaverbatim}


\section{算法说明}

必须实现的算法有:

\begin{enumerate}
    \item SVD 迭代 $\Rightarrow$ \verb|SVDMethod/SVDMethod|
\end{enumerate}

其中 SVD 分解中 2x2 的实现参考了:
\url{https://www.math.ucla.edu/~cffjiang/research/svd/svd.pdf},
在此表示感谢。

\section{运行结果}

也可在 \verb|homeworks/reports/data/report_8_output.txt| 中查看。

\verbatiminput{data/report_8_output.txt}

\newpage

\end{document}