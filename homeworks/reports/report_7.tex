\documentclass{article}
\usepackage[UTF8]{ctex}
\usepackage{float,indentfirst,verbatim,fancyhdr,graphicx,listings,longtable,amsmath, amsfonts,amssymb}

\textheight 23.5cm \textwidth 15.8cm
\topmargin -1.5cm \oddsidemargin 0.3cm \evensidemargin -0.3cm

\title{数值代数实验报告 7}
\author{马天开}

\begin{document}
\maketitle

\section{问题描述}

\subsection{求实对称三对角阵的全部特征值和特征向量}

\subsubsection{}用 C++ 编制利用过关 Jacobi 方法求实对称三对角阵全部特征值和特征向量的通用子程序

\subsubsection{}利用你所编制的子程序求 50, 60, 70, 80, 90, 100 阶矩阵

\[
A=\begin{bmatrix}
\;4 & 1 & 0 & 0 & \cdots & 0 \;\\
\;1 & 4 & 1 & 0 & \cdots & 0 \;\\
\;0 & 1 & 4 & 1 & \cdots & 0 \;\\
\;\vdots & & \ddots & \ddots & \ddots \;\\
\;0 & \cdots & 0 & 1 & 4 & 1\;\\
\;0 & 0 & \cdots & 0 & 1 & 4\;
\end{bmatrix}
\]
的全部特征值和特征向量。

\subsection{求实对称三对角阵的指定特征值及对应的特征向量}

\subsubsection{}用 C++ 编制先利用二分法求实对称三对角阵指定特征值,再利用反幂法求对应特征向量的通用子程序

\subsubsection{利用你所编制的子程序求 100 阶矩阵}
\[
A=\begin{bmatrix}
\;2 & -1 & 0 & 0 & \cdots & 0 \;\\
\;-1 & 2 & -1 & 0 & \cdots & 0 \;\\
\;0 & -1 & 2 & -1 & \cdots & 0 \;\\
\;\vdots & & \ddots & \ddots & \ddots \;\\
\;0 & \cdots & 0 & -1 & 2 & -1\;\\
\;0 & 0 & \cdots & 0 & -1 & 2\;
\end{bmatrix}
\]
的最大和最小特征值及对应的特征向量

\section{算法说明}

必须实现的算法有:

\begin{enumerate}
    \item 经典 Jacobi 方法 & 过关 Jacobi 方法 $\Rightarrow$ \verb|JacobiMethod/JacobiMethod|
    \item 二分法 分解 $\Rightarrow$ \verb|BisectMethod/BisectMethod|
    \item 反幂法 $\Rightarrow$ \verb|PowerIteration/RevPowerIteraion|
\end{enumerate}

调整了大部分命名、输出格式、循环结果等。

\section{运行结果}

也可在 \verb|homeworks/reports/data/report_7_output.txt| 中查看。

\verbatiminput{data/report_7_output.txt}

\newpage

\end{document}